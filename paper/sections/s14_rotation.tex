\subsection{Motivation}

\Cref{sec:consistency} assumed a fixed validator set $V$.
In practice, validators join or leave between epochs, keys must be rotated to
maintain forward secrecy, and a static committee creates long-range attack
exposure.
This section formalizes epoch transitions and proves that safety and liveness
are preserved across committee changes.

\subsection{Epoch and Validator Set Model}

\begin{definition}[Epoch]
\[
  \mathsf{Epoch}_e = (V_e,\; \mathit{vk}_e,\; T^e_{\mathrm{start}},\;
  T^e_{\mathrm{end}},\; \mathit{seed}_e),
\]
where $V_e$ is the validator set, $\mathit{vk}_e$ the aggregated threshold
verification key, and $\mathit{seed}_e$ the VDF beacon seed.
The BFT parameters are $n_e = |V_e|$, $f_e < n_e/3$, and
$t_e = \lfloor 2n_e/3 \rfloor + 1$.
\end{definition}

\subsection{Handoff MPC and Key Rotation}

Key rotation proceeds in three in-epoch phases.

\paragraph{Phase~1: DKG (first two-thirds of epoch~$e$).}
$V_{e+1}$ runs a Distributed Key Generation protocol (e.g., Feldman
VSS~\cite{feldman1987practical}) producing secret shares distributed among
$V_{e+1}$ and public key $\mathit{vk}_{e+1}$.

\paragraph{Phase~2: Handoff Certificate (last third of epoch~$e$).}
\begin{equation}
  \mathit{HC}_e = \mathsf{TSig}_{V_e}\!\left(
    H\!\left(\mathit{vk}_{e+1} \;\|\; e{+}1 \;\|\;
    S_{\mathrm{last},e}\right)\right).
\end{equation}
$\mathit{HC}_e$ certifies $\mathit{vk}_{e+1}$ and binds it to the last
finalized state of epoch~$e$.

\paragraph{Phase~3: Epoch transition.}
The first batch of epoch~$e{+}1$ includes $\mathit{HC}_e$.
Clients verify $\mathit{HC}_e$ before accepting signatures under
$\mathit{vk}_{e+1}$.
Members of $V_e$ delete their shares of $\mathit{sk}_e$ (best-effort forward
secrecy).

\subsection{Safety Across Rotation}

\begin{theorem}[Epoch Transition Safety]\label{thm:rotation-safety}
Under \cref{thm:safety}'s assumptions applied to both $V_e$ and $V_{e+1}$,
the same nullifier $\mathit{sn}$ cannot be finalized in two different epochs.
\end{theorem}
\begin{proof}[Proof sketch]
A transaction finalized in epoch~$e$ inserts $\mathit{sn}$ into
$S_{\mathrm{last},e}.\mathsf{SpentSetRoot}$.
The first batch of epoch~$e{+}1$ chains to $H(S_{\mathrm{last},e})$ (via
$\mathit{prev\_hash}$) and $V_{e+1}$ checks non-membership against
$S_{\mathrm{last},e}.\mathsf{SpentSetRoot}$.
An adversary cannot present a forged $S_{\mathrm{last},e}$ because
$\mathit{HC}_e$ is bound to the genuine last state and is unforgeable under
TSig unforgeability (\cref{sec:consistency}).
\end{proof}

\subsection{Liveness Across Rotation}

\begin{theorem}[Epoch Transition Liveness]\label{thm:rotation-liveness}
If DKG completes and $\mathit{HC}_e$ is produced within epoch~$e$, then
epoch~$e{+}1$ begins with a fully operational committee within $O(\Delta)$
after $T^e_{\mathrm{end}}$.
\end{theorem}

\subsection{Long-Range Attack Resistance}

\begin{definition}[Weak Subjectivity Window]
Clients do not accept state assertions older than $W$ epochs without an
out-of-band trusted checkpoint.
\end{definition}

\begin{theorem}[Long-Range Resistance]\label{thm:longrange}
Rewriting state from epoch $e < \mathit{current} - W$ requires one of:
(i) forging $\mathit{HC}_{e-1}$ (ruled out by TSig unforgeability), or
(ii) solving $T_{\mathrm{beacon}}$ sequential VDF squarings per epoch
(sequential hardness of VDF), or (iii) presenting a forged checkpoint
(ruled out by the weak subjectivity assumption).
\end{theorem}

\subsection{Validator Churn Tolerance}

\begin{definition}[Churn Rate]
$\mathit{churn}_e = |V_e \mathbin{\triangle} V_{e+1}| / n$.
\end{definition}

\begin{theorem}[Churn Safety Bound]\label{thm:churn}
If $|V_e \cap V_{e+1}|$ contains at least $t_e$ honest validators, then safety
is preserved across the epoch boundary.
\end{theorem}

\begin{corollary}
Maximum safe churn per epoch is $\mathit{churn}_e \leq 1/3$.
Full committee replacement in a single epoch transition is unsafe.
\end{corollary}
