\subsection{Adversarial Capabilities}

We consider a PPT adversary $\mathcal{A}$ that:
\begin{itemize}
  \item controls an arbitrary fraction of non-validator network nodes;
  \item schedules and delays messages arbitrarily before GST (the Global
        Stabilization Time of the DLS partial-synchrony
        model~\cite{dwork1988consensus});
  \item adaptively corrupts up to $f_e < n_e/3$ validators within each
        epoch~$e$; and
  \item is computationally bounded (PPT).
\end{itemize}

$\mathcal{A}$ may attempt to: forge coin proofs, double-spend a coin,
inflate value, deanonymize senders, or violate state consistency during
epoch transitions.
$\mathcal{A}$ may also attempt sustained spam/DoS
(\cref{sec:dos}).

\subsection{Cryptographic Assumptions}

The following hardness assumptions are used throughout; each is stated where
it is first required.

\begin{itemize}
  \item \textbf{DL}: Discrete logarithm in $\mathbb{G}$ is hard.
  \item \textbf{CRH}: $H$ is collision resistant.
  \item \textbf{PRF}: $\mathsf{PRF}_s(\cdot)$ is a secure pseudorandom
        function.
  \item \textbf{sRSA}: Strong RSA assumption in $\mathbb{Z}_N^*$ (underlies
        accumulator security).
  \item \textbf{KS}: Knowledge soundness of the SNARK proof system
        (Groth16: holds under the knowledge-of-exponent assumption, which is
        non-falsifiable but standard in pairing-based SNARK analyses; PLONK:
        holds in the algebraic group model).
  \item \textbf{ZK}: Zero-knowledge of the SNARK proof system.
  \item \textbf{TSig-UF}: Threshold signature unforgeability (BLS TSig under
        DL).
  \item \textbf{VDF-Seq}: Sequentiality of the VDF (modular squaring in a
        group of unknown order).
\end{itemize}

\subsection{Network Model}

We use the partial synchrony model of Dwork--Lynch--Stockmeyer
\cite{dwork1988consensus}: after an unknown GST, all messages arrive within
a known bound $\Delta$.
Before GST, \emph{safety} is maintained; \emph{liveness} is guaranteed only
after GST.

\subsection{Validator Corruption Bound}

In epoch $e$, there are $n_e$ validators; at most $f_e < n_e/3$ may be
Byzantine.
Finality threshold: $t_e = \lfloor 2n_e/3 \rfloor + 1$.
Byzantine validators may send conflicting messages, withhold signatures, or
attempt to equivocate; honest validators follow the protocol exactly.

\subsection{Security Goals}

We target the following properties, proved in the sections indicated.

\begin{description}
  \item[Correctness] (\cref{sec:consistency}): Honest execution produces
        accepted transactions.
  \item[Unforgeability] (\cref{sec:atomicity}): No PPT adversary can spend
        a coin without knowing its witness.
  \item[Value Conservation]: No PPT adversary can inflate coin values.
  \item[Double-Spend Resistance] (\cref{sec:consistency}): The same nullifier
        cannot be finalized twice.
  \item[Anonymity] (\cref{sec:committee}): An adversary cannot distinguish
        which of two candidate coins produced a challenge transaction.
  \item[Non-Malleability]: A valid transaction cannot be transformed to steal
        value without knowing witness material.
  \item[Safety] (\cref{sec:consistency}): No two honest verifiers accept
        conflicting finalized state.
  \item[Liveness] (\cref{sec:consistency}): After GST, valid transactions are
        finalized within $O(T_{\mathrm{beacon}} + \kappa\Delta)$.
  \item[Atomic Transition] (\cref{sec:atomicity}): ACC and SpentSetRoot update
        jointly or not at all.
\end{description}

\subsection{Out of Scope}

We explicitly do not claim:
\begin{itemize}
  \item \emph{Full censorship resistance}: we target only weak fairness via VDF
        ordering and attributable misbehavior (\cref{sec:consistency}).
  \item \emph{Bitcoin-equivalent trustlessness}: Groth16 requires a trusted
        setup; the RSA accumulator requires a trusted modulus generation.
        Eliminating both requires setup-free proofs and class-group
        accumulators, which we defer to future work.
  \item \emph{Network-layer DoS immunity}: a state-level adversary with
        unlimited bandwidth can flood the physical network; this is outside the
        scope of this protocol.
\end{itemize}
