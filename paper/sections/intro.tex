\section{Introduction}
\label{sec:intro}

\subsection{Problem Statement}

Since Bitcoin introduced the blockchain as a global append-only ledger,
essentially all decentralized payment systems have inherited this architectural
decision.
The ledger simultaneously serves three functions: establishing a canonical
transaction order, preventing double-spending by recording all past nullifiers,
and providing a verifiable audit trail for coin validity.
This coupling is not architecturally necessary.
The three functions can be decomposed---and the audit trail, which drives
ledger growth, can be eliminated entirely if coin validity is established by
cryptographic proof rather than chain membership.

Existing privacy-preserving systems such as Zcash~\cite{zcash2022protocol}
demonstrate that double-spend prevention can be reduced to a nullifier set and
coin validity to zero-knowledge proofs~\cite{ben2014zerocash}.
Yet Zcash retains a blockchain as the substrate for these structures, requiring
every full node to store and synchronize the complete transaction history.
The resulting state growth is linear in transaction count, creating long-run
pressure toward centralization by increasing storage, bandwidth, and initial
synchronization cost for participants.

\subsection{This Work}

We ask: \emph{can a payment system achieve the security properties of
Zcash---double-spend resistance, full transaction privacy, value
conservation---without any global transaction ledger?}

We answer affirmatively, subject to standard cryptographic hardness assumptions
and a partial synchrony network model~\cite{dwork1988consensus}.
Our construction, \textsc{Snarc Axiom} (\textsc{Axiom} for short), maintains only two global state objects whose
sizes are independent of transaction history:
\begin{itemize}
  \item \textbf{ACC}: a dynamic RSA accumulator over coin commitments---constant
        256~bytes regardless of coin set size; and
  \item \textbf{SpentSetRoot}: a Sparse Merkle Tree root over nullifier
        hashes---constant 32~bytes, with the underlying set checkpointed and
        archived separately.
\end{itemize}
These objects are finalized by a BFT committee producing threshold signatures
over batched state transitions, yielding finalization certificates that replace
ledger entries.
A VDF beacon~\cite{pietrzak2019vdf,wesolowski2019vdf} constrains leader
discretion in transaction ordering without requiring a trusted clock.

\subsection{Contributions}

\begin{itemize}
  \item \textbf{C1 --- Ledgerless payment protocol.}
    We define \textsc{Snarc Axiom} and prove that it achieves double-spend resistance,
    value conservation, and full transaction anonymity without a global
    transaction ledger as the canonical state repository, relying instead on
    compact state objects and finalization certificates.

  \item \textbf{C2 --- Formal security model.}
    We provide a complete threat model, six cryptographic security definitions,
    and proof sketches under standard assumptions: discrete logarithm hardness,
    collision resistance, PRF security, RSA strong assumption, and SNARK
    knowledge soundness~\cite{groth2016size,gabizon2019plonk}.

  \item \textbf{C3 --- Atomic state transition with BFT finality.}
    We define a deterministic State Transition Function $\mathsf{Apply}(S,B)
    \to S'$ (\cref{sec:atomicity}) and prove it is atomic under the hybrid
    BFT+VDF consistency sublayer~\cite{castro1999pbft}, replacing the
    blockchain's role in ordering and finalizing state updates.
    The threshold signature covers both the batch content and the resulting
    state roots jointly, making partial observation impossible at the protocol
    level.

  \item \textbf{C4 --- Permissionless committee selection.}
    We formalize VRF-based sortition~\cite{gilad2017algorand,rfc9381} with
    stake-weighted admission (\cref{sec:committee}), proving sybil
    non-amplification, expected committee size bounds, and compatibility with
    the BFT safety threshold.

  \item \textbf{C5 --- Spam and DoS resistance.}
    We define a two-tier mempool architecture with three composable admission
    mechanisms---stake tickets, PoW-lite hashcash, and VRF tickets---and prove
    bounded verification work and spam cost dominance (\cref{sec:dos}).

  \item \textbf{C6 --- Concrete instantiation and benchmarks.}
    We instantiate \textsc{Axiom} with Groth16 over BLS12-381, RSA-2048
    accumulator, BLS threshold signatures, and ECVRF (RFC~9381), providing
    circuit constraint estimates, latency decomposition, throughput analysis,
    and a phased implementation roadmap (\cref{sec:bench}).
\end{itemize}

\subsection{Comparison to Related Work}

\textsc{Axiom}'s cryptographic core shares its nullifier-and-ZK-proof structure
with Zerocash~\cite{ben2014zerocash} and Zcash~\cite{zcash2022protocol}.
The key architectural departure is the elimination of the blockchain as state
substrate: Zcash uses the chain as the nullifier set carrier and coin existence
log, while \textsc{Axiom} replaces both with \textsf{ACC} and
\textsf{SpentSetRoot} finalized by BFT threshold certificates.
This direction parallels Utreexo~\cite{dryja2019utreexo}, which replaces
Bitcoin's UTXO chain history with a compact accumulator---but applied to a
privacy-preserving, ledgerless setting.
The BFT finality and partial synchrony model follows
Castro--Liskov~\cite{castro1999pbft} and
Dwork--Lynch--Stockmeyer~\cite{dwork1988consensus}.
Permissionless committee selection extends Algorand's VRF
sortition~\cite{gilad2017algorand} with adaptive sizing and epoch rotation.
VDF-based ordering follows Pietrzak~\cite{pietrzak2019vdf} and
Wesolowski~\cite{wesolowski2019vdf}; for a unified treatment see
Boneh~et~al.~\cite{boneh2018vdfsurvey}.

\subsection{Paper Organization}

Sections~\ref{sec:consistency}--\ref{sec:bench} present the protocol and
analysis in order: the hybrid BFT+VDF consistency sublayer, accumulator
atomicity, spam resistance, epoch rotation, permissionless committee selection,
and concrete benchmarks.
\Cref{sec:related} surveys related work and \cref{sec:conclusion} concludes.
