This section addresses responsible deployment of \textsc{Snarc Axiom},
independent of the cryptographic security proofs in preceding sections.
Cryptographic soundness is a necessary but not sufficient condition for safe
deployment of a financial protocol.
The policies below bound the impact of any failure---whether cryptographic,
implementation, or operational---and define the conditions under which the
system escalates, freezes, or migrates.

\subsection{Deployment Tiers}

\textsc{Snarc Axiom} is defined across three deployment tiers with strictly
increasing security requirements and value exposure.

\paragraph{Tier~0 --- Researchnet (v1).}
The protocol described in this paper.
Instantiation: Groth16 + RSA-2048 + stake-VRF BFT.
Coins carry no monetary value.
Purpose: academic analysis, benchmark collection, circuit validation, and
adversarial testing.
\emph{No real-value assets are issued or redeemable under Tier~0.}

\paragraph{Tier~1 --- Testnet (v2).}
Universal-setup or setup-free proof system (PLONK/Halo2).
Hybrid PoW-lite + stake committee admission.
Value caps enforced at the protocol level (see \cref{sec:policy-caps}).
Independent security audit required before any value-bearing token launch.
Bug bounty active.

\paragraph{Tier~2 --- Mainnet (v3).}
Setup-free proof system (STARKs or equivalent).
Post-quantum signatures (e.g., Dilithium/Falcon) for committee TSig.
PQ-resistant accumulator (hash-based state proof or class-group accumulator).
Multi-audit and formal verification of critical modules.
Full regulatory and legal review per jurisdiction.
\emph{Only Tier~2 is suitable for production monetary value.}

\subsection{Cryptographic Agility}

All cryptographic components are treated as plug-in modules behind stable
interfaces:

\begin{itemize}
  \item \textbf{Proof system}: $\Pi.\{\mathsf{Setup}, \mathsf{Prove},
        \mathsf{Verify}\}$ — replaceable without changing the ZK relation
        $\mathcal{R}$ constraints.
  \item \textbf{Accumulator}: $\mathsf{ACC}.\{\mathsf{Update},
        \mathsf{VerifyMem}\}$ — RSA-2048 $\to$ class-group $\to$
        hash-based, migrated via epoch transition (\cref{sec:rotation}).
  \item \textbf{Threshold signature}: BLS12-381 $\to$ lattice-based TSig,
        swapped during a scheduled epoch rotation.
  \item \textbf{Hash / PRF}: BLAKE3/Poseidon $\to$ SHA-3 family or
        SPHINCS-compatible; parameter-upgradeable.
\end{itemize}

A migration between tiers is executed as a coordinated epoch transition
(\cref{sec:rotation}) with a mandatory overlap period of at least $W$~epochs
(weak subjectivity window) during which both the old and new proof systems
are accepted.

\subsection{Threat Escalation and Freeze Policy}

Three escalation levels are defined based on observed or credible threat
signals.

\paragraph{Level~1 --- Yellow (Monitor).}
Trigger: published theoretical attack, unconfirmed anomaly, or unusual
on-chain pattern detected by monitoring.
Response: increase audit frequency; convene security committee within 24~h;
prepare freeze parameters; no operational change yet.

\paragraph{Level~2 --- Orange (Restrict).}
Trigger: credible proof-of-concept exploit reported, or anomalous state
growth/nullifier collision rate exceeding $3\sigma$ baseline.
Response: (i) halt \emph{MintTx} and \emph{BurnTx} immediately;
(ii) reduce per-epoch batch cap $M_{\max}$ to 10\% of normal;
(iii) notify all validators and auditors;
(iv) begin emergency migration preparation.

\paragraph{Level~3 --- Red (Full Freeze).}
Trigger: confirmed exploit, double-finalization observed, or trusted-setup
compromise credibly demonstrated.
Response: (i) halt all \emph{SpendTx}, \emph{MintTx}, and
\emph{BurnTx}; (ii) only \emph{read-only} state queries remain live;
(iii) publish incident report within 6~h; (iv) initiate emergency migration
or coordinated shutdown.

The freeze condition is enforced by the BFT committee via a threshold-signed
\emph{HaltTx}: any $t$-of-$n$ validator quorum can issue a halt, and no
validator will sign new batch proposals until a validated \emph{ResumeTx}
(also requiring $t$-of-$n$ threshold authorization) is produced.
\emph{HaltTx and ResumeTx do not confer any ability to reassign coin
ownership; they only suspend and resume state transitions under threshold
authorization.}
Neither transaction can move funds, alter accumulator witnesses, or modify
the SpentSet; the $\mathsf{Apply}$ function is simply not called until
\emph{ResumeTx} is finalized.

\subsection{Value Caps}
\label{sec:policy-caps}

Any Tier~1 deployment must enforce the following, with the enforcement
mechanism explicitly specified:

\begin{itemize}
  \item \textbf{Per-transaction cap} ($v \leq V_{\max}^{\mathit{tx}}$):
        enforced \emph{in-circuit} as a range constraint over the committed
        value field.
        The ZK proof is unsatisfiable for any $v > V_{\max}^{\mathit{tx}}$;
        no trusted party can override this.

  \item \textbf{Per-epoch issuance cap}
        ($\sum_{\mathit{MintTx} \in B_k} v_i \leq
        V_{\max}^{\mathit{epoch}}$):
        enforced as a \emph{committee validation rule}.
        Validators maintain a per-epoch issuance counter in the batch
        proposal; honest validators reject any proposal exceeding the cap.
        The counter is included in the threshold-signed finalization object
        $C_k$ and is therefore auditable.

  \item \textbf{Total outstanding supply cap}
        ($\sum_{\mathit{cm} \in \mathsf{ACC}} v_{\mathit{cm}} \leq
        V_{\max}^{\mathit{total}}$):
        enforced as a \emph{committee validation rule} backed by an
        accumulator-derived supply counter in $S_k$.
        As with the epoch cap, honest validators reject proposals that would
        push the counter above $V_{\max}^{\mathit{total}}$.

  \item \textbf{Redemption gate}: BurnTx is disabled by default;
        enabled only after explicit audit sign-off per deployment.
        The gate is a committee rule, not a circuit constraint, because
        enabling it requires no change to the ZK circuit.
\end{itemize}

\subsection{Audit and Bug Bounty Requirements}

\paragraph{Tier~0.}
Internal review only.
Open-source release with explicit ``experimental, no value'' label.

\paragraph{Tier~1.}
Minimum two independent external security audits of:
(i) the ZK circuit and relation $\mathcal{R}$;
(ii) the BFT + VDF finality sublayer;
(iii) the admission and spam-resistance layer.
Bug bounty active from day~1 of public testnet; severity-tiered rewards.

\paragraph{Tier~2.}
All Tier~1 requirements plus:
formal verification of the \textsf{Apply} state transition function
(\cref{thm:atomicity}) using a proof assistant (e.g., Coq, Lean);
hardware wallet and side-channel audit;
jurisdiction-specific legal review.

\subsection{Quantum Threat Timeline and Migration Trigger}

The v1 cryptographic components (Groth16, RSA-2048, BLS12-381) are not
post-quantum secure.
A sufficiently large fault-tolerant quantum computer would break them via
Shor's algorithm.
Current consensus among cryptographers places such a machine at 10--20+
years away for cryptographically relevant scale, though this estimate
carries significant uncertainty.

Migration triggers are defined as \emph{externally verifiable events}, not
subjective forecasts, to ensure objective escalation without reliance on
any single party's judgment:

\begin{enumerate}
  \item \textbf{Standards milestone}: NIST formally deprecates RSA-2048 or
        elliptic-curve cryptography for near-term use (e.g., via a FIPS
        revision or sunset advisory with a fixed end-of-life date).

  \item \textbf{Community consensus}: A reproducible break or significant
        speedup against discrete-log or factoring is documented in a
        peer-reviewed publication at a major venue (IEEE S\&P, CCS, Crypto,
        Eurocrypt) and independently confirmed by at least one other research
        group.

  \item \textbf{Practical break}: A publicly reproducible demonstration
        breaks a concrete instance of BLS12-381 DL, RSA-2048 factoring, or
        the Groth16 proof system at any key size used in the deployment.
\end{enumerate}

On any trigger, Level~2 (Orange) is declared within 24~hours, and the
migration to a setup-free, PQ-resistant configuration (Tier~2) begins under
the epoch rotation protocol (\cref{sec:rotation}).
Trigger conditions are defined as externally verifiable events (standards
milestones or publicly reproducible breaks), not subjective forecasts.

\subsection{Responsibility Allocation}

This paper describes a cryptographic protocol design.
Responsibility allocation for production deployments is as follows:

\begin{itemize}
  \item \textbf{Protocol authors}: specification correctness; honest
        disclosure of limitations (this section and \cref{sec:conclusion}).
  \item \textbf{Implementers}: correctness of the implementation against the
        specification; side-channel hardening; wallet security.
  \item \textbf{Operators}: audit compliance; value-cap enforcement;
        incident response execution; regulatory compliance.
  \item \textbf{Users}: understanding the risk tier of the deployment they
        use; not exceeding their own risk tolerance.
\end{itemize}

\noindent\textit{No deployment of \textsc{Snarc Axiom} at any tier carries
an implied guarantee of security or suitability for any particular use.
The protocol is provided as a research artifact under open-source terms.
Use at own risk.}
