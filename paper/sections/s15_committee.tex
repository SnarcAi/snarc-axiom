\subsection{Motivation}

\Cref{sec:rotation} introduced epoch-based committee rotation under a
stake-weighted admission model.
This section extends that model to fully permissionless participation via
VRF-based sortition, proving sybil resistance and compatibility with the BFT
safety threshold.

\subsection{Design Goals}

\begin{itemize}
  \item \textbf{G9 (Open Entry).} Any node depositing minimum stake may
        participate.
  \item \textbf{G10 (Unpredictability).} Committee membership for epoch~$e$
        cannot be predicted before epoch~$e{-}1$ ends.
  \item \textbf{G11 (Proportionality).} Selection probability is proportional
        to stake.
  \item \textbf{G12 (Sybil Resistance).} Splitting stake across $k$ identities
        yields no additional advantage.
\end{itemize}

\subsection{Beacon Seed Chain}

\begin{definition}[Beacon Seed Chain]
\[
  \mathit{seed}_0 = H(\text{``AXIOM\_GENESIS''} \| \mathit{pp}),\quad
  \mathit{seed}_{e+1} = H\!\left(\mathit{seed}_e \;\|\;
    \mathsf{VDF}(\mathit{seed}_e, T_{\mathrm{vdf}}) \;\|\;
    H(B_{\mathrm{last},e})\right).
\]
\end{definition}

\begin{lemma}[Seed Unpredictability]\label{lem:seed}
Under VRF pseudorandomness and VDF sequentiality, the advantage of any PPT
adversary controlling $f < n/3$ validators in predicting $\mathit{seed}_{e+1}$
at the start of epoch~$e$ is negligible.
\end{lemma}
\begin{proof}[Proof sketch]
$\mathsf{VDF}(\mathit{seed}_e, T_{\mathrm{vdf}})$ requires $T_{\mathrm{vdf}}$
sequential steps and cannot be precomputed; honest epoch activity contributes
fresh entropy via $H(B_{\mathrm{last},e})$.
By VRF pseudorandomness, knowing $\mathit{seed}_e$ does not allow predicting
which nodes will produce high-ranking VRF outputs.
\end{proof}

\subsection{Sortition Protocol}

\begin{definition}[Sortition]\label{def:sortition}
In epoch~$e$, each node~$v$ with stake $\mathit{stake}_v$ computes:
\[
  (y_v, \pi_v) = \mathsf{VRF.Eval}(\mathit{sk}_v,\;
  \mathit{seed}_e \| \text{``COMMITTEE''} \| e).
\]
Node~$v$ is \emph{selected} iff:
\[
  \frac{H(y_v)}{2^\lambda} \;\leq\; \tau(e) \cdot
  \frac{\mathit{stake}_v}{\mathit{stake\_total}_e}.
\]
Selected nodes broadcast $(\mathit{pk}_v, y_v, \pi_v, \mathit{stake}_v,
w_{\mathit{stake}})$, where $w_{\mathit{stake}}$ is a Merkle proof of stake.
\end{definition}

\begin{theorem}[Expected Committee Size]\label{thm:committee-size}
$\mathbb{E}[|V_e|] = \tau(e) \cdot (\mathit{stake\_active} /
\mathit{stake\_unit})$.
Setting $\tau(e) = n_{\mathit{target}} \cdot \mathit{stake\_unit} /
\mathit{stake\_active}$ yields $\mathbb{E}[|V_e|] = n_{\mathit{target}}$.
\end{theorem}

\begin{theorem}[Sybil Non-Amplification]\label{thm:sybil}
An adversary with total stake $S_{\mathit{adv}}$ splitting across $k$
identities has the same expected committee seats as a single identity with
stake $S_{\mathit{adv}}$.
\end{theorem}
\begin{proof}
For $k$ identities each with stake $S_{\mathit{adv}}/k$:
\[
  \Pr[\text{at least one selected}]
  = 1 - \prod_{i=1}^k \!\left(1 - \tau \cdot
    \frac{S_{\mathit{adv}}/k}{\mathit{stake\_total}}\right)
  \;\underset{k\to\infty}{\longrightarrow}\;
  \tau \cdot \frac{S_{\mathit{adv}}}{\mathit{stake\_total}},
\]
which equals the single-identity selection probability.
\end{proof}

\subsection{Adaptive Committee Sizing}

\begin{definition}[Adaptive $\tau$]
\[
  \tau(e{+}1) = \tau(e) \cdot \frac{n_{\mathit{target}}}{|V_e|},
  \quad
  \tau(e) \geq \tau_{\min} = \frac{(3f_{\mathit{target}}+1) \cdot
  \mathit{stake\_unit}}{\mathit{stake\_active}}.
\]
$\tau_{\min}$ ensures $\mathbb{E}[|V_e|] \geq 3f_{\mathit{target}}+1$ at all
times.
\end{definition}

\subsection{Bootstrap and Genesis Committee}

The first epoch uses a pre-announced genesis committee of $k \geq 3f+1$ early
stake depositors.
This is not a trusted setup: genesis members receive no special protocol
privileges beyond epoch~0.
From epoch~1 onward, sortition (\cref{def:sortition}) fully controls
committee membership.

\subsection{Honest Disclosure}

Stake concentration creates plutocracy risk; mitigation options (quadratic
weighting, per-validator stake caps) are deployment choices not specified here.
``Nothing-at-stake'' behavior (signing multiple forks) is deterred by slashing
(\cref{sec:dos}, P4).
Low overall stake participation can compress committee size toward $\tau_{\min}$,
reducing liveness margin.
